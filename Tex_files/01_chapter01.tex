\section{Analog multipliers theory}

\subsection{Introduction}
\begin{frame} % display current subsection in subindex
\tableofcontents[currentsubsection]
\end{frame}


% new slide
\begin{frame}
	\frametitle{Analog multipliers working principles}
	\textbf{Analog Multiplier} circuit that performs the product between two signals. Spectral components from a certain frequency band are moved to another by means of intrinsic non-linear behaviour (modulation). The working principle is below:
	\begin{figure}[H]
	\centering
	\scalebox{0.78}
	{
		\begin{circuitikz} 
			\ctikzset{tripoles/mos style/arrows,bipoles/length=1cm}
			\draw
			(0,0) node[mixer] (m) {}
			(m.1) to[short,-] ++(-.5,0) node[left]{$x_1(t)= A_1 \cos(\omega_1 t + \varphi_1)$}
			(m.2) to[short,-] ++(0,-.5) node[below]{$x_2(t) = A_2 \cos(\omega_2 t + \varphi_2)$}
			(m.3) to[short,-] ++(.5,0) node[inputarrow]{} (.5,0) node[right=3mm]{$y(t)= \frac{A_1 A_2}{2} \cos(\omega_{LF} t + \varphi_{LF}) +\frac{A_1 A_2}{2} \cos(\omega_{HF} t + \varphi_{HF})$}
			(m.1) node[inputarrow] {} 
			(m.2) node[inputarrow,rotate=90] {};
		\end{circuitikz}
	}		
	\label{Mixer2}	
	\end{figure}
	Two \textbf{new} out-of-phase spectral component at output:
	\begin{description}
		\item[Down-converted] at lower frequency: $\omega_{LF}= |\omega_1-\omega_2|<\omega_1,\omega_2 $
		\item[Up-converted] at higher frequency: $\omega_{HF}= |\omega_1+\omega_2|>\omega_1,\omega_2$
	\end{description}


	
\end{frame}


% new slide
\begin{frame}
	\frametitle{Analog multipliers working principles}
	Mixers are \emph{bi-direction} three-port. At each port a signal is associated
	\begin{description}
		\item[RF]  radio frequency component, high frequency signal;
		\item[IF] intermediate frequency component, low frequency signal;
		\item[LO]  local oscillator (pump), provided by external source.
	\end{description}
	Depending on input/output signal configuration we can have \emph{downconverting} or \emph{upconverting} modulators.
	
		\begin{figure}[H]
			\centering
			\scalebox{0.9}
			{
				\begin{circuitikz} 
					\ctikzset{tripoles/mos style/arrows,bipoles/length=1cm}
					\draw
					(0,0) node[mixer] (m) {}
					(m.1) to[short,-] ++(-.5,0) node[left]{RF}
					(m.2) to[short,-] ++(0,-.5) node[below]{LO}
					(m.3) to[short,-] ++(.5,0) node[inputarrow]{} (.5,0) node[right=3mm]{IF}
					(m.1) node[inputarrow] {} 
					(m.2) node[inputarrow,rotate=90] {};
					\draw
					(6,0) node[mixer] (m) {}
					(m.1) to[short,-] ++(-.5,0) node[left]{IF}
					(m.2) to[short,-] ++(0,-.5) node[below]{LO}
					(m.3) to[short,-] ++(.5,0) node[inputarrow]{} (6.5,0) node[right=3mm]{RF}
					(m.1) node[inputarrow] {} 
					(m.2) node[inputarrow,rotate=90] {};
				\end{circuitikz}
			}	
			\caption{Working configuration: downconversion (right), upconversion (left).}	
			\label{Mixerdown}	
		\end{figure}

		
	
\end{frame}

% new slide
\begin{frame}
	\frametitle{Mixer figures of merit}
	To qualify the mixer operation some figures of merit are defined (downconversion mixer):
	\begin{itemize}
		\item \textbf{Conversion gain}
		\begin{equation}
			A_{conv}=\frac{P_{IF}}{P_{RF}} \notag 
		\end{equation}
		whose behaviour is linear in log scale:
		\begin{equation}
			P_{IF}|_{dB_{m}} = A_{conv}|_{dB} + P_{RF}|_{dB_{m}}+30dB \notag
		\end{equation}
		\item \textbf{1dB compression point} defines the input power level for which the difference between the compressed and non-compressed output power is 1dB. It is due to gain saturation caused by harmonic distortion:
		\begin{equation}
			P_{IF}|_{-1dB} = P_{IF}|_{dB_{m},ideal} -1dB \notag
		\end{equation} 
	\end{itemize}
\end{frame}

% new slide
\begin{frame}
	\frametitle{Mixer figures of merit}
	\begin{itemize}
		\item \textbf{Third order distortion} most important output spurious contribution due to gain compression:
		\begin{align}
			i_D(t) &= I_D|_{V_{GS}}+av_{gs}(t)+bv_{gs}^2(t)+cv_{gs}^3(t)+\dots \notag \\
			& \propto av_{gs}(t)+bv_{gs}^2(t)+\frac{3}{4}c\cos(\omega_0 t) +\frac{1}{4}c\cos(3\omega_0 t) \notag
		\end{align}
		Even order distortion gives additional DC offset, odd order distortion gives always in-band unwanted components.
		\item \textbf{Third order intermodulation} it happens with two-tone input signals because of frequency intermodulation. Most important contribution comes from IM\textsubscript{3}: $m=\pm2,\pm1$ and $n=\pm2,\pm1$. Given $f_{RF1} = f_0$ and $f_{RF2}=f_0+\delta f$:
		\begin{equation}
			f_{IF,IM3}|_{\substack{m=2\\n=-1}} = 2f_{RF1} - f_{RF2} -f_{LO}= f_{IF} - \delta f \notag
		\end{equation} 
	\end{itemize}
\end{frame}

\subsection{Analysis of a Gilbert cell based multiplier}

% new slide
\begin{frame} % display current subsection in subindex
\tableofcontents[currentsubsection]
\end{frame}

\begin{frame}
	\frametitle{Gilbert cell overview}
	A Gilbert-cell mixer is an active double-balanced analog multiplier. This topology provides:
	\begin{itemize}
		\item Reasonable conversion gain;
		\item Good rejection of input frequency components at the output port, high linearity;
		\item Good isolation between ports;
		\item Integrability in CMOS technology.
	\end{itemize}
	We can recognise \textbf{four main blocks}: bias net, gain stage, mixing stage and load. 
\end{frame}

%new slide
\begin{frame}
\frametitle{Gilbert cell circuit analysis - Bias net}
\begin{figure}[H] 
	\centering
	\subfloat[][\emph{Bias net}]{\scalebox{0.6}{\begin{circuitikz}
				\ctikzset{tripoles/mos style/arrows,bipoles/length=1cm}
				\ctikzset{bipoles/capacitor/height=0.5}
				\ctikzset{bipoles/capacitor/width=0.1}
				%M2
				\draw (0,0) to[Tnmos,mirror,n=M2] (0,2);
				\draw (M2.source) node[left=3mm,above=3mm]{$M2$};
				\draw (M2.gate)[right] |- (M2.drain);
				\draw (M2.gate) to[short,-*] (2.5,1) node[right]{to $G_1$};
				\draw (M2.source) to[short] (0,0) node[sground]{};
				\draw (0,6) -- (-2,6) to[C=$C_{bias}$] (-2,5) node[sground]{};
				%M5
				\draw (M2.drain) to[Tnmos,mirror,n=M5] (0,4.5);
				\draw (M5.source) node[left=3mm,above=3mm]{$M5$};
				\draw (M5.gate)[right] |- (M5.drain);
				\draw (0,4) -- (-2,4) to[C=$C_{bias}$] (-2,3) node[sground]{};
				\draw (M5.gate) to[R, l_=$R_1$] (2,2.3) to[short,-*] (2.5,2.3) node[right]{to $G_3$};
				\draw (M5.gate) to[R=$R_1$] (2,3.7) to[short,-*] (2.5,3.7) node[right]{to $G_4$};
				%R2 R4
				\draw (M5.drain) to[R=$R_2$,n=R2] (0,6.3) to[R=$R_4$] (0,7.1) to[short,-*] (0,7.5) node[above]{$V_{dd}$};
				\draw (0,5.7) to[short] (0.7,5.7) to[R,l_=$R_3$] (2,5) to[short,-*] (2.5,5) node[right]{to $G_6$,$G_9$};
				\draw (0,5.7) to[short] (0.7,5.7) to[R=$R_3$] (2,6.4) to[short,-*] (2.5,6.4) node[right]{to $G_7$,$G_8$};
	\end{circuitikz}}} \quad
	\subfloat[][\emph{Gilbert cell}]{\scalebox{0.6}{\begin{circuitikz}
				\ctikzset{tripoles/mos style/arrows,bipoles/length=1cm}
				\ctikzset{bipoles/capacitor/height=0.5}
				\ctikzset{bipoles/capacitor/width=0.1}
				%drawing MOS
				\draw (0,0) to[Tnmos,n=M1] (0,2)
				(M1.source) node[right=3mm, above=3mm]{$M1$};
				\draw (M1.gate) to[short,-*] (-1,1);
				
				\draw (0,2) to[R,l=$R_S$] (-2,2)
				to[Tnmos,n=M3] (-2,4)
				(M3.source) node[right=3mm, above=3mm]{$M3$};
				
				\draw (0,2) to[R,l_=$R_S$] (2,2) to[Tnmos,mirror,n=M4] (2,4)
				(M4.source) node[left=3mm, above=3mm]{$M4$};
				
				\draw (-2,4) -- (-3,4)
				to[Tnmos,n=M6] (-3,5.5)
				(M6.source) node[right=3mm, above=3mm]{$M6$};
				
				\draw (-2,4) -- (-1,4) to[Tnmos,mirror,n=M7] (-1,5.5)
				(M7.source) node[left=3mm, above=3mm]{$M7$};
				
				\draw (2,4) -- (1,4) to[Tnmos,n=M8] (1,5.5)
				(M8.source) node[right=3mm, above=3mm]{$M8$};
				
				\draw (2,4) -- (3,4) to[Tnmos,mirror,n=M9] (3,5.5)
				(M9.source) node[left=3mm, above=3mm]{$M9$};
				
				%drawing VLO-
				\draw (M7.gate) -- (M8.gate);
				\draw (M7.gate) -| (0,4.5);
				\draw (0,4.5) to[C=$C_{signal}$] (0,3) to[short,-*] (0,3) node[below]{$V_{LO}-$};
				
				%drawing RL and out connections
				\draw (M6.drain) -- (-3,6) to[R=$R_L$,n=RL1] (-3,7) -- (-3,7.5);
				\draw (M9.drain) --(3,6) to[R=$R_L$,n=RL2] (3,7) -- (3,7.5);
				\draw (-1,5.5) -- (3,6);
				\draw (1,5.5) -- (-3,6);
				
				%Vdd and ground
				\draw (-3,7.5) node[above=3mm,right=3cm]{$V_{dd}$} -- (3,7.5);
				\draw (M1.source) -- (0,0) node[sground]{};
				
				% VLO+-
				\draw (M6.gate) -| (-4,4.75) to[short,-*] (-4,4.75) node[left]{to $G9$};
				%-| (-4,4.7) to[short,-*] (-4,3) node[left]{to $G_9$};
				\draw (M9.gate) -| (4,4.7) to[C=$C_{signal}$] (4,3) to[short,-*] (4,3) node[below]{$V_{LO}+$};
				
				% VRF+-
				\draw (M3.gate) -| (-3,2) to[C, l_=$C_{signal}$] (-3,1) to[short,-*] (-3,0.5) node[left]{$V_{RF}+$};
				\draw (M4.gate) -| (3,2) to[C, l_=$C_{signal}$] (3,1) to[short,-*] (3,0.5) node[left]{$V_{RF}-$};
				%\draw (M4.gate) -- (3,3) to[short,-*] (3,3) node[right]{$V_{RF}-$};
				
				%Out nodes
				\draw (-3, 6) to[short,*-*] (-4, 6) node[left]{$V_{out}+$};
				\draw (3, 6) to[short,*-*] (4, 6)node[right]{$V_{out}-$};
	\end{circuitikz}}}
	\caption{Full schematic}
	\label{fig:TdomaniDFT}
\end{figure}
\end{frame}

\begin{frame}
	\frametitle{Gilbert cell circuit analysis - Bias net}
	Ideally current mirroring only dependent on \textbf{geometrical} parameters. However, in real circuits:
	\begin{itemize}
		\item To have good current source long channel required, since $r_o=1/\lambda I_0\propto L$. Using short channel devices i\textsubscript{D} more dependent on $\lambda$ and tolerances;
		\item Due to fabrication precision, MOSFET's parameters may vary:
		\begin{equation}
		\frac{I_0}{I_{REF}} \simeq 1 + \frac{\Delta K_n}{K_n}+2\frac{\Delta V_{th}}{V_{od}} \notag
		\end{equation}
	\end{itemize}
	Wide circuits, temperature gradients and small overdrive voltage significantly induce mirroring errors.
\end{frame}

\begin{frame}
	\frametitle{Gilbert cell circuit analysis - Gain stage}
	\begin{columns} [c]
		\column{.45\textwidth}
		Thanks to source degeneration we reduce bias dependency and  increase \textbf{linearity} (high order harmonics suppression).
		\begin{equation}
		\label{eq_degenGain}
		G_{m,eq} = \frac{g_m}{1+g_m R_S} \notag
		\end{equation}
		\column{.5\textwidth}
		\begin{figure}[H]
			\centering
			\scalebox{0.8}{
			\begin{circuitikz}
				\ctikzset{tripoles/mos style/arrows,bipoles/length=1cm}
				%I_0 and RS
				\draw (0,0) to[short,i<=$I_0$] (0,1);
				\draw (0,1) to[R,l_=$R_S$] (-1,1) -| (-1.5,1.5) to[Tnmos,n=M3] (-1.5,2) to[twoport,l=$LO_{stage}$] (-1.5,4);
				\draw (M3.source) node[right=3mm, above=3mm]{$M3$};
				\draw (0,1) to[R,l=$R_S$] (1,1) -| (1.5,1.5) to[Tnmos,n=M4,mirror] (1.5,2) to[twoport,l=$LO_{stage}$] (1.5,4);
				\draw (M4.source) node[left=3mm, above=3mm]{$M4$};
				\draw (M3.gate) -| (-2.5,1.7) to[short,-*] (-2.5,1.75) node[below]{$V_{in1}$};
				\draw (M4.gate) -| (2.5,1.7) to[short,-*] (2.5,1.75) node[below]{$V_{in2}$};
				%Vx Vy
				\draw (M3.drain) -| (-1,2.3) to[short,-*] (-1,2.3) node[right]{$V_x$};
				\draw (M4.drain) -| (1,2.3) to[short,-*] (1,2.3) node[left]{$V_x$};
			\end{circuitikz}
			}
			\caption{Gain stage.}
			\label{fig:GainStage}
		\end{figure}
	\end{columns}
\end{frame}


\begin{frame}
	\frametitle{Gilbert cell circuit analysis - Mixing stage}
	\begin{figure} [H]
		\centering
		\scalebox{0.6}{
		\begin{circuitikz}
			\ctikzset{tripoles/mos style/arrows}
			\draw (-2,4) -- (-3,4)
			to[Tnmos,n=M6] (-3,5.5)
			(M6.source) node[right=3mm, above=5mm]{$M6$};
			
			\draw (-2,4) -- (-1,4) to[Tnmos,mirror,n=M7] (-1,5.5)
			(M7.source) node[left=3mm, above=5mm]{$M7$};
			
			\draw (2,4) -- (1,4) to[Tnmos,n=M8] (1,5.5)
			(M8.source) node[right=3mm, above=5mm]{$M8$};
			
			\draw (2,4) -- (3,4) to[Tnmos,mirror,n=M9] (3,5.5)
			(M9.source) node[left=3mm, above=5mm]{$M9$};
			
			%drawing VLO-
			\draw (M7.gate) -- (M8.gate);
			\draw (M7.gate) -| (0,4.5);
			\draw (0,4.5) to[short,-*] (0,3) node[below]{$V_{LO}-$};
			
			%Out nodes
			\draw (-3, 6) to[short,*-*] (-4, 6) node[left]{$V_{out}+$};
			\draw (3, 6) to[short,*-*] (4, 6)node[right]{$V_{out}-$};
			\draw (M6.drain) to[short] (-3,7);
			\draw (M6.gate) -| (-4,3) to[short,-*] (-4,3) node[below]{$V_{LO}+$};
			\draw (M7.drain) -- (3,6);
			\draw (M8.drain) -- (-3,6);
			\draw (M9.drain) to[short] (3,7);
			\draw (M9.gate) -| (4,3) to[short,-*] (4,3) node[below]{$V_{LO}+$};
			\draw (-2,3) -- (-2,4)
			(2,3) -- (2,4);
		\end{circuitikz}
		}
		\caption{Mixing stage}
		\label{fig:MixStage}
	\end{figure}
	Mixing stage is made up of switching-like driven transistors that acts as \textbf{pulse amplitude modulators}. % PAM is different from PWM!!!!!!!!
	A complementary sinusoidal signal is applied to the gates of each pair (V\textsubscript{LO}), large enough to ensure the \textbf{abrupt switching} of one MOSFET whereas the other must be kept in saturation (not in  triode!).
\end{frame}

\begin{frame}
	\frametitle{Gilbert cell circuit analysis - Mixing stage}
	Compromise in driving signal required:
	\begin{itemize}
		\item small LO signals produce slower switching speed and reduces efficiency (as common mode signal at the output);
		\item large LO signals drive the device in triode (spikes and unwanted feed-through reducing gain).
	\end{itemize}
	Switching speed is related to MOSFET transition frequency:
	\begin{equation}
	f_T \propto  V_{od}/L \notag
	\end{equation}
	Then fast devices are obtained with \textbf{short channel} length and \textbf{large overdrive}. Larger devices introduce parasitic capacitance, detrimental for the conversion efficiency and speed. Maximum working frequency should be below $f_T/10$.
\end{frame}

\begin{frame}
	\frametitle{Gilbert cell circuit analysis - Mixing stage}
	Mixing stage is non linear, however given the square wave current flowing through each device:
	\begin{equation}
	i_D(t)=I_{pk}\Bigg(\frac{1}{2}-\frac{2}{\pi} \sum_{n=1,3,5 \dots}^{} \frac{1}{n}\sin(n\omega t)\Bigg) \notag
	\end{equation} 
	one has the instantaneous gain:
	\begin{equation}
	\label{eq_SwitchGain}
	A_c|_{switch} = \frac{2}{\pi} \notag
	\end{equation}
	associated to first harmonic, ideally not depending on device properties. For real devices the actual gain is lower.
\end{frame}

\begin{frame}
	\frametitle{Gilbert cell circuit analysis - Load stage}
	Two resistors are employed as loads. They are less noisy than transistors (Flicker noise at low frequency), less reliable for what concern tolerances though. They are required to provide enough gain to the stage, if necessary active loads can be employed to maximize gain.
\end{frame}

\begin{frame}
	\frametitle{Gilbert cell circuit analysis - Conversion gain}
	The circuit is non-linear, therefore it is not possible to define a gain. It can be demonstrated that the static \textbf{conversion gain} for the cell is:
	\begin{equation}
	\label{eq:ConvGain}
	A_{vC} = \frac{V_{IF,rms}}{V_{RF,rms}} =  \frac{2}{\pi}\frac{R_L}{\frac{1}{g_{m3,4}}+R_S} \simeq \frac{2}{\pi}\frac{R_L}{R_S} \notag
	\end{equation}
	Some important facts hold:
	\begin{itemize}
		\item in first approximation the conversion gain does not depend on the amplitude of the LO signal;
		\item both LO and RF components are rejected at the output;
	\end{itemize}  
\end{frame}