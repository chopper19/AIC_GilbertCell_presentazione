\section{Circuit design}

\subsection{Design by hand of a down-converting Gilbert cell}
\begin{frame} % display current subsection in subindex
\tableofcontents[currentsubsection]
\end{frame}

\begin{frame}
\frametitle{Design specifications}
\begin{block}{Technology and purposes}
Technology used is MOSIS AMI 0.6$\mu$m. Loose constraint of $\sim 1mm$ is taken for maximum gate width. This technology is old fashioned and not the best for this purpose.
\end{block}

\begin{block}{Supply voltage}
Supply voltage is chosen $5V$ 
\end{block}

\begin{block}{Frequencies}
Single frequency down-converting mixer with: \\ f\textsubscript{RF}=110 MHz, f\textsubscript{LO}=100 MHz, f\textsubscript{IF}=10 MHz.
\end{block}
\begin{block}{Conversion Gain}
Voltage conversion gain is chosen A\textsubscript{vC}=4. All transistors are supposed in saturation.
\end{block}
\end{frame}

\begin{frame}
\frametitle{Used parameters}
\begin{table} [h]
\label{tab:specs}
\caption{}
\centering	
\begin{tabular}{llcc} 
\toprule 
Parameter & Name			& Value 	& Unit \\ 
\midrule
A\textsubscript{vC} & Voltage Conversion gain & 4 & \\
V\textsubscript{DD} & Supply voltage &	5 & V		\\
I\textsubscript{0} & Cell biasing current & 5 & mA \\
V\textsubscript{th0} & n-MOS threshold w.o. body effect& 0.709 &V		\\ 
K\textsubscript{n} & n-MOS physical parameter& 116 & $\mu$A/V\textsuperscript{2}\\
I\textsubscript{dss} & Maximum channel current density & 466 & $\mu$A/$\mu$m \\
$\phi_P$ & Fermi potential & 0.7 & V \\
$\gamma_B$ & Body effect coefficient & 0.5 & V \\
L\textsubscript{min} & Technology minimum length & 0.6 & $\mu$m \\
\bottomrule 
\end{tabular}	
\end{table}	
\end{frame}

\begin{frame}
\frametitle{Reference schematic}
\begin{figure}[H] 
\centering
\subfloat[][\emph{Bias net}]{\scalebox{0.7}{\begin{circuitikz}
			\ctikzset{tripoles/mos style/arrows,bipoles/length=1cm}
			\ctikzset{bipoles/capacitor/height=0.5}
			\ctikzset{bipoles/capacitor/width=0.1}
			%M2
			\draw (0,0) to[Tnmos,mirror,n=M2] (0,2);
			\draw (M2.source) node[left=3mm,above=3mm]{$M2$};
			\draw (M2.gate)[right] |- (M2.drain);
			\draw (M2.gate) to[short,-*] (2.5,1) node[right]{to $G_1$};
			\draw (M2.source) to[short] (0,0) node[sground]{};
			\draw (0,6) -- (-2,6) to[C=$C_{1}$] (-2,5) node[sground]{};
			%M5
			\draw (M2.drain) to[Tnmos,mirror,n=M5] (0,4.5);
			\draw (M5.source) node[left=3mm,above=3mm]{$M5$};
			\draw (M5.gate)[right] |- (M5.drain);
			\draw (0,4) -- (-2,4) to[C=$C_{2}$] (-2,3) node[sground]{};
			\draw (M5.gate) to[R, l_=$R_1$] (2,2.3) to[short,-*] (2.5,2.3) node[right]{to $G_3$};
			\draw (M5.gate) to[R=$R_1$] (2,3.7) to[short,-*] (2.5,3.7) node[right]{to $G_4$};
			%R2 R4
			\draw (M5.drain) to[R=$R_2$,n=R2] (0,6.3) to[R=$R_4$] (0,7.1) to[short,-*] (0,7.5) node[above]{$V_{dd}$};
			\draw (0,5.7) to[short] (0.7,5.7) to[R,l_=$R_3$] (2,5) to[short,-*] (2.5,5) node[right]{to $G_6$,$G_9$};
			\draw (0,5.7) to[short] (0.7,5.7) to[R=$R_3$] (2,6.4) to[short,-*] (2.5,6.4) node[right]{to $G_7$,$G_8$};
\end{circuitikz}}}
\subfloat[][\emph{Gilbert mixer}]{\scalebox{0.7}{\begin{circuitikz}
\ctikzset{tripoles/mos style/arrows,bipoles/length=1cm}
\ctikzset{bipoles/capacitor/height=0.5}
\ctikzset{bipoles/capacitor/width=0.1}
%drawing MOS
\draw (0,0) to[Tnmos,n=M1] (0,2)
(M1.source) node[right=3mm, above=3mm]{$M1$};
\draw (M1.gate) to[short,-*] (-1,1);

\draw (0,2) to[R,l_=$R_S$] (-2,2)
to[Tnmos,n=M3] (-2,4)
(M3.source) node[right=3mm, above=3mm]{$M3$};

\draw (0,2) to[R,l=$R_S$] (2,2)
to[Tnmos,mirror,n=M4] (2,4)
(M4.source) node[left=3mm, above=3mm]{$M4$};

\draw (-2,4) -- (-3,4)
to[Tnmos,n=M6] (-3,5.5)
(M6.source) node[right=3mm, above=3mm]{$M6$};

\draw (-2,4) -- (-1,4) to[Tnmos,mirror,n=M7] (-1,5.5)
(M7.source) node[left=3mm, above=3mm]{$M7$};

\draw (2,4) -- (1,4) to[Tnmos,n=M8] (1,5.5)
(M8.source) node[right=3mm, above=3mm]{$M8$};

\draw (2,4) -- (3,4) to[Tnmos,mirror,n=M9] (3,5.5)
(M9.source) node[left=3mm, above=3mm]{$M9$};

%drawing VLO-
\draw (M7.gate) -- (M8.gate);
\draw (M7.gate) -| (0,4.5);
\draw (0,4.5) to[C=$C_{signal}$] (0,3) to[short,-*] (0,3) node[below]{$V_{LO}-$};

%drawing RL and out connections
\draw (M6.drain) -- (-3,6) to[R=$R_L$,n=RL1] (-3,7) -- (-3,7.5);
\draw (M9.drain) --(3,6) to[R=$R_L$,n=RL2] (3,7) -- (3,7.5);
\draw (-1,5.5) -- (3,6);
\draw (1,5.5) -- (-3,6);

%Vdd and ground
\draw (-3,7.5) node[above=3mm,right=3cm]{$V_{dd}$} -- (3,7.5);
\draw (M1.source) -- (0,0) node[sground]{};

% VLO+-
\draw (M6.gate) -| (-4,4.75) to[short,-*] (-4,4.75) node[left]{to $G9$};
%-| (-4,4.7) to[short,-*] (-4,3) node[left]{to $G_9$};
\draw (M9.gate) -| (4,4.7) to[C=$C_{signal}$] (4,3) to[short,-*] (4,3) node[below]{$V_{LO}+$};

% VRF+-
\draw (M3.gate) -| (-3,2) to[C, l_=$C_{signal}$] (-3,1) to[short,-*] (-3,0.5) node[left]{$V_{RF}+$};
\draw (M4.gate) -| (3,2) to[C, l_=$C_{signal}$] (3,1) to[short,-*] (3,0.5) node[left]{$V_{RF}-$};
%\draw (M4.gate) -- (3,3) to[short,-*] (3,3) node[right]{$V_{RF}-$};

%Out nodes
\draw (-3, 6) to[short,*-*] (-4, 6) node[left]{$V_{out}+$};
\draw (3, 6) to[short,*-*] (4, 6)node[right]{$V_{out}-$};
\end{circuitikz}}}
\caption{Bias net and Gilbert Mixer schematic}
\label{fig:Gilb_designHand_schem}
\end{figure}
\end{frame}

\begin{frame}
\frametitle{Current bias design}
\begin{columns}
\column[]{0.5\textwidth} To have good mirroring this must be satisfied:\begin{gather}
V_{GS1} = V_{VGS2} \notag \\
V_{DS1} = V_{DS2} \notag \\
\left(\frac{W}{L}\right)_1 = \left(\frac{W}{L}\right)_2 \notag 
\end{gather}
The values we impose are:
\begin{align}
&I_{ref} = I_0 = 5 mA  \notag\\
&V_{od1}=V_{od2}=0.4 V  \notag\\
&L_1 = L_2 = 3 L_{mim} = 1.8 \mu m \notag
\end{align}
\column[]{0.5\textwidth}
\begin{figure}
\scalebox{1}{
\begin{circuitikz}
\ctikzset{tripoles/mos style/arrows,bipoles/length=1cm}
\ctikzset{bipoles/capacitor/height=0.5}
\ctikzset{bipoles/capacitor/width=0.1}
%M2
\draw (-2,0) to[Tnmos,mirror,n=M2] (-2,1.5);
\draw (-2,3) to[I=$I_{ref}$] (M2.drain);
\draw (M2.source) node[sground]{};
\draw (M2.source) node[left=3mm, above=3mm]{$M2$};
%M1
\draw (1,0) to[Tnmos,n=M1] (1,1.5);
\draw (M1.source) node[sground]{};
\draw (M1.gate) -- (M2.gate) |- (M2.drain);
%Current short
\draw (1,3) to[short,i>=$I_0$] (M1.drain);
\draw (M1.source) node[right=3mm, above=3mm]{$M1$};
\draw (1.5,0) to[open, v=$V_{DS1}$] (1.5,1.5);
\draw (M1.source) to[open,v^=$V_{GS1}$] (M1.gate);
\end{circuitikz}}
\caption{Biasing current mirror}
\label{fig:BiasCurrent2}
\end{figure}
\end{columns}
\end{frame}

\begin{frame}
\frametitle{Current bias design}
\begin{columns}
\column[]{0.5\textwidth} 
...we derive gate width and gate bias
\begin{gather}
W_1 = \frac{2I_0}{K_n V_{od1}^2}L_1 = 969.8 \mu m \notag\\ 
V_{th1}=V_{th2}=V_{th0} \notag 
\end{gather}
(no body effect)
\begin{gather}
V_{GS2}=V_{DS2}= V_{th0}+\sqrt{\frac{2I_0L_1}{K_n W_1 }} = 1.1 V \nonumber
\end{gather}
\column[]{0.5\textwidth}
\begin{figure}
\scalebox{1}{
\begin{circuitikz}
\ctikzset{tripoles/mos style/arrows,bipoles/length=1cm}
\ctikzset{bipoles/capacitor/height=0.5}
\ctikzset{bipoles/capacitor/width=0.1}
%M2
\draw (-2,0) to[Tnmos,mirror,n=M2] (-2,1.5);
\draw (-2,3) to[I=$I_{ref}$] (M2.drain);
\draw (M2.source) node[sground]{};
\draw (M2.source) node[left=3mm, above=3mm]{$M2$};
%M1
\draw (1,0) to[Tnmos,n=M1] (1,1.5);
\draw (M1.source) node[sground]{};
\draw (M1.gate) -- (M2.gate) |- (M2.drain);
%Current short
\draw (1,3) to[short,i>=$I_0$] (M1.drain);
\draw (M1.source) node[right=3mm, above=3mm]{$M1$};
\draw (1.5,0) to[open, v=$V_{DS1}$] (1.5,1.5);
\draw (M1.source) to[open,v^=$V_{GS1}$] (M1.gate);
\end{circuitikz}}
\caption{Biasing current mirror}
\label{fig:BiasCurrent3}
\end{figure}
\end{columns}
\end{frame}

%new slide
\begin{frame}
\frametitle{Load resistance design}

\begin{columns}
\column[]{0.5\textwidth} 
I\textsubscript{0} is evenly split into M3 and M4.

In each LO only one transistor per time is on.

In order to have enough output swing we decided to drop across the load:
\begin{gather}
V_{R_L} = \frac{1}{3}\cdot V_{DD} \nonumber
\end{gather}
This leads to the choosing of load
\begin{gather}
R_L = \frac{\frac{1}{3}V_{DD}}{I_0/2} = 667\Omega \nonumber
\end{gather}

\column[]{0.5\textwidth}
\begin{figure}
\scalebox{0.7}{
\begin{circuitikz}
\ctikzset{tripoles/mos style/arrows,bipoles/length=1cm}

\draw (-2,4) -- (-3,4)
to[Tnmos,n=M6] (M6.drain) to[R,l=$R_L$] (-3,7) -- (3,7)
(M6.source) node[right=3mm, above=3mm]{$M6$};

\draw (-2,4) -- (-1,4) to[Tnmos,mirror,n=M7] (M7.drain) to[R,l=$R_L$] (-1,7)
(M7.source) node[left=3mm, above=4mm]{$M7$};

\draw (2,4) -- (1,4) to[Tnmos,n=M8] (M8.drain) to[R,l=$R_L$] (1,7)
(M8.source) node[right=3mm, above=3mm]{$M8$};

\draw (2,4) -- (3,4) to[Tnmos,mirror,n=M9] (M9.drain) to[R,l=$R_L$] (3,7)
(M9.source) node[left=3mm, above=3mm]{$M9$};

%drawing VLO-
\draw (M7.gate) -- (M8.gate);
\draw (M7.gate) -| (0,4) to[short,-*] (0,4);

%Out nodes
\draw (M6.gate) -| (-5,4.65)  to[short,-*] (-5,4.65);
\draw (M9.gate) -| (5,4.65)  to[short,-*] (5,4.65);
\draw (-2,3) -- (-2,4)
(2,3) -- (2,4);
\draw (0,7) node[above]{$V_{DD}$};
\draw (-2,0) node[sground]{} to[open, v^=$V_{SB6}$] (M6.source) to[open, v^=$V_{GS6}$] (M6.gate);
\draw (-2.5,4) to[open,v=$V_{DS6}$] (-2.5,5);
\draw (-4,5) to[open,v^=$V_{RL}$] (-4,7);
\end{circuitikz}
}
\caption{Load voltage drop on mixing stage's equivalent circuit.}
\label{fig:Load resistors}
\end{figure}
\end{columns}
\end{frame}

%new slide
\begin{frame}
\frametitle{Gain stage design}
\begin{columns}
\column[]{0.5\textwidth} 
R\textsubscript{S} is chosen in order to increase stage linearity without provoking a too large voltage drop.
\begin{gather}
R_S = 10\Omega \notag\\ 
V_{R_S}=\frac{R_S}{I_0/2} = 25 mV \notag
\end{gather}

\column[]{0.5\textwidth}
\begin{figure}
\scalebox{0.7}{
\begin{circuitikz}
\ctikzset{tripoles/mos style/arrows,bipoles/length=1cm}
%I_0 and RS
\draw (0,-1) node[sground]{};
\draw (0,-1) to[Tnmos,n=M1] (0,0);
\draw (M1.gate) to[short,-*] (M1.gate)node[right=3mm, above=3mm]{$M1$};
\draw (0,0) to[short,i_<=$I_0$] (0,1);
\draw (0,1) to[short,-] (-0.2,1) to[R,l_=$R_S$] (-2,1) -| (-2.5,2) to[Tnmos,n=M3] (-2.5,3) to[short,-] (-2.5,3.5) to[twoport,l=$LO_{stage}$] (-2.5,4.5);
\draw (M3.source) node[right=3mm, above=3mm]{$M3$};
\draw (0,1) to[short,-] (0.2,1) to[R,l=$R_S$] (2,1) -| (2.5,2) to[Tnmos,n=M4,mirror] (2.5,3) to[short,-] (2.5,3.5) to[twoport,l=$LO_{stage}$] (2.5,4.5);
\draw (M4.source) node[left=3mm, above=3mm]{$M4$};
\draw (M3.gate) -| (-4,2.5) to[short,-*] (-4,2.5) node[below]{};
\draw (M4.gate) -| (4,2.5) to[short,-*] (4,2.5) node[below]{};
%voltage arrows
\draw (M1.source) to[open, v^=$V_{GS1}$] (M1.gate);
\draw (-1.5,-1.7) to[open, v^=$V_{SB3}$] (-2.5,1);
\draw (M3.source) to[open, v^=$V_{GS3}$] (M3.gate);
\draw (-2,1.7) to[open, v=$V_{DS3}$] (-2,3);
\draw (0,0.9) to[open, v^=$V_{RS}$] (-2,0.9);
\end{circuitikz}
}
\caption{Gain stage}
\label{fig:Gain stage Rs}
\end{figure}
\end{columns}
\end{frame}

\begin{frame}
\frametitle{Gain stage design}
\begin{columns}
\column[]{0.5\textwidth} 
Given the previous specification $A_{vC}=4$ and conversion gain expression
\begin{gather}
A_{vC} = \frac{V_{IF,rms}}{V_{RF,rms}} =  \frac{2}{\pi}\frac{R_L}{\frac{1}{g_{m3,4}}+R_S} \notag
\end{gather}
We can then evaluate transconductance and gate width (provided $L=3L_{min}=1.8\mu m$)
\begin{align}
&g_{m3} = \frac{\pi}{2}\frac{1}{\frac{R_L}{A_{vC}}-R_S}=10 mS \notag\\
&W_3 = g_{m3}^2\frac{2L_{3}}{K_nI_0} = 624 \mu m \notag
\end{align}

\column[]{0.5\textwidth}
\begin{figure}
\scalebox{0.7}{
\begin{circuitikz}
\ctikzset{tripoles/mos style/arrows,bipoles/length=1cm}
%I_0 and RS
\draw (0,-1) node[sground]{};
\draw (0,-1) to[Tnmos,n=M1] (0,0);
\draw (M1.gate) to[short,-*] (M1.gate)node[right=3mm, above=3mm]{$M1$};
\draw (0,0) to[short,i_<=$I_0$] (0,1);
\draw (0,1) to[short,-] (-0.2,1) to[R,l_=$R_S$] (-2,1) -| (-2.5,2) to[Tnmos,n=M3] (-2.5,3) to[short,-] (-2.5,3.5) to[twoport,l=$LO_{stage}$] (-2.5,4.5);
\draw (M3.source) node[right=3mm, above=3mm]{$M3$};
\draw (0,1) to[short,-] (0.2,1) to[R,l=$R_S$] (2,1) -| (2.5,2) to[Tnmos,n=M4,mirror] (2.5,3) to[short,-] (2.5,3.5) to[twoport,l=$LO_{stage}$] (2.5,4.5);
\draw (M4.source) node[left=3mm, above=3mm]{$M4$};
\draw (M3.gate) -| (-4,2.5) to[short,-*] (-4,2.5) node[below]{};
\draw (M4.gate) -| (4,2.5) to[short,-*] (4,2.5) node[below]{};
%voltage arrows
\draw (M1.source) to[open, v^=$V_{GS1}$] (M1.gate);
\draw (-1.5,-1.7) to[open, v^=$V_{SB3}$] (-2.5,1);
\draw (M3.source) to[open, v^=$V_{GS3}$] (M3.gate);
\draw (-2,1.7) to[open, v=$V_{DS3}$] (-2,3);
\draw (0,0.9) to[open, v^=$V_{RS}$] (-2,0.9);
\end{circuitikz}
}
\caption{Gain stage}
\label{fig:Gain stage gm}
\end{figure}
\end{columns}
\end{frame}

\begin{frame}
\frametitle{Gain stage design}
\begin{columns}
\column[]{0.5\textwidth} 
M3 is subjected to body effect, thus:
\begin{gather}
V_{SB3} = V_{DS1}+V_{R_S} \notag\\
V_{th3} = V_{th0}+\gamma_B\big(\sqrt{2\phi_P + V_{SB3}}-\sqrt{2\phi_P}\big) = 0.957 V \notag
\end{gather}
This leads to an higher gate-source voltage:
\begin{align}
&V_{od3}=\sqrt{\frac{I_0}{K_n W_3/L_3}} = 0.353 V \notag\\
&V_{GS3} = V_{th3}+V_{od3} = 1.31 V \notag
\end{align}

\column[]{0.5\textwidth}
\vspace{3cm}
\begin{figure}
\scalebox{0.6}{
\begin{circuitikz}
\ctikzset{tripoles/mos style/arrows,bipoles/length=1cm}
%I_0 and RS
\draw (0,-1) node[sground]{};
\draw (0,-1) to[Tnmos,n=M1] (0,0);
\draw (M1.gate) to[short,-*] (M1.gate)node[right=3mm, above=3mm]{$M1$};
\draw (0,0) to[short,i_<=$I_0$] (0,1);
\draw (0,1) to[short,-] (-0.2,1) to[R,l_=$R_S$] (-2,1) -| (-2.5,2) to[Tnmos,n=M3] (-2.5,3) to[short,-] (-2.5,3.5) to[twoport,l=$LO_{stage}$] (-2.5,4.5);
\draw (M3.source) node[right=3mm, above=3mm]{$M3$};
\draw (0,1) to[short,-] (0.2,1) to[R,l=$R_S$] (2,1) -| (2.5,2) to[Tnmos,n=M4,mirror] (2.5,3) to[short,-] (2.5,3.5) to[twoport,l=$LO_{stage}$] (2.5,4.5);
\draw (M4.source) node[left=3mm, above=3mm]{$M4$};
\draw (M3.gate) -| (-4,2.5) to[short,-*] (-4,2.5) node[below]{};
\draw (M4.gate) -| (4,2.5) to[short,-*] (4,2.5) node[below]{};
%voltage arrows
\draw (M1.source) to[open, v^=$V_{GS1}$] (M1.gate);
\draw (-1.5,-1.7) to[open, v^=$V_{SB3}$] (-2.5,1);
\draw (M3.source) to[open, v^=$V_{GS3}$] (M3.gate);
\draw (-2,1.7) to[open, v=$V_{DS3}$] (-2,3);
\draw (0,0.9) to[open, v^=$V_{RS}$] (-2,0.9);
\end{circuitikz}
}
\caption{Gain stage}
\label{fig:Gain stage body}
\end{figure}
\end{columns}
\end{frame}

\begin{frame}
\frametitle{Gain stage design}
\begin{columns}
\column[]{0.5\textwidth} 
We impose to have half the supply voltage to drop on the gain and current mirror stages. From here:
\begin{gather}
V_{DS3} = \frac{1}{2}V_{DD}-V_{DS1} = 1.4 V \notag
\end{gather}
Since this is higher than overdrive, M3 saturation is guaranteed if on. Based on this we evaluate the gate bias voltage that keeps underneath stages saturated:
\begin{gather}
V_{G3}=V_{GS3} +V_{R_{S}}+V_{DS1} = 2.425 V \notag
\end{gather}
\column[]{0.5\textwidth}
\begin{figure}
\scalebox{0.7}{
\begin{circuitikz}
\ctikzset{tripoles/mos style/arrows,bipoles/length=1cm}
%I_0 and RS
\draw (0,-1) node[sground]{};
\draw (0,-1) to[Tnmos,n=M1] (0,0);
\draw (M1.gate) to[short,-*] (M1.gate)node[right=3mm, above=3mm]{$M1$};
\draw (0,0) to[short,i_<=$I_0$] (0,1);
\draw (0,1) to[short,-] (-0.2,1) to[R,l_=$R_S$] (-2,1) -| (-2.5,2) to[Tnmos,n=M3] (-2.5,3) to[short,-] (-2.5,3.5) to[twoport,l=$LO_{stage}$] (-2.5,4.5);
\draw (M3.source) node[right=3mm, above=3mm]{$M3$};
\draw (0,1) to[short,-] (0.2,1) to[R,l=$R_S$] (2,1) -| (2.5,2) to[Tnmos,n=M4,mirror] (2.5,3) to[short,-] (2.5,3.5) to[twoport,l=$LO_{stage}$] (2.5,4.5);
\draw (M4.source) node[left=3mm, above=3mm]{$M4$};
\draw (M3.gate) -| (-4,2.5) to[short,-*] (-4,2.5) node[below]{};
\draw (M4.gate) -| (4,2.5) to[short,-*] (4,2.5) node[below]{};
%voltage arrows
\draw (M1.source) to[open, v^=$V_{GS1}$] (M1.gate);
\draw (-1.5,-1.7) to[open, v^=$V_{SB3}$] (-2.5,1);
\draw (M3.source) to[open, v^=$V_{GS3}$] (M3.gate);
\draw (-2,1.7) to[open, v=$V_{DS3}$] (-2,3);
\draw (0,0.9) to[open, v^=$V_{RS}$] (-2,0.9);
\end{circuitikz}
}
\caption{Gain stage}
\label{fig:Gain stage saturation}
\end{figure}
\end{columns}
\end{frame}

\begin{frame}
\frametitle{Mixing stage design}
\begin{columns}
\column[]{0.5\textwidth}
A very small overdrive is needed to foster fast transitions on LO stage. 
\begin{gather}
V_{od6} = 150mV  \notag
\end{gather}
Small overdrives produce large gate width. 

To limit this, the minimum gate length was chosen for LO stage. Then we can derive gate width:
\begin{gather}
L_6 = L_{min}  \notag\\
W_6= \frac{I_0 L_6}{K_n V_{od6}^2} = 1.1mm\notag
\end{gather}
\column[]{0.5\textwidth}
\begin{figure}
\scalebox{0.5}{
\begin{circuitikz}
\ctikzset{tripoles/mos style/arrows,bipoles/length=1cm}

\draw (-2,4) -- (-3,4)
to[Tnmos,n=M6] (M6.drain) to[R,l=$R_L$] (-3,7) -- (3,7)
(M6.source) node[right=3mm, above=3mm]{$M6$};

\draw (-2,4) -- (-1,4) to[Tnmos,mirror,n=M7] (M7.drain) to[R,l=$R_L$] (-1,7)
(M7.source) node[left=3mm, above=4mm]{$M7$};

\draw (2,4) -- (1,4) to[Tnmos,n=M8] (M8.drain) to[R,l=$R_L$] (1,7)
(M8.source) node[right=3mm, above=3mm]{$M8$};

\draw (2,4) -- (3,4) to[Tnmos,mirror,n=M9] (M9.drain) to[R,l=$R_L$] (3,7)
(M9.source) node[left=3mm, above=3mm]{$M9$};

%drawing VLO-
\draw (M7.gate) -- (M8.gate);
\draw (M7.gate) -| (0,4) to[short,-*] (0,4);

%Out nodes
\draw (M6.gate) -| (-5,4.65)  to[short,-*] (-5,4.65);
\draw (M9.gate) -| (5,4.65)  to[short,-*] (5,4.65);
\draw (-2,3) -- (-2,4)
(2,3) -- (2,4);
\draw (0,7) node[above]{$V_{DD}$};
\draw (-2,0) node[sground]{} to[open, v^=$V_{SB6}$] (M6.source) to[open, v^=$V_{GS6}$] (M6.gate);
\draw (-2.5,4) to[open,v=$V_{DS6}$] (-2.5,5);
\draw (-4,5) to[open,v^=$V_{RL}$] (-4,7);
\end{circuitikz}
}
\caption{Gain stage}
\label{fig:Mixing stage stage saturation}
\end{figure}
\end{columns}
\end{frame}

\begin{frame}
\frametitle{Mixing stage design}
\begin{columns}
\column[]{0.5\textwidth}
Taking into account body effect:
\begin{align}
&V_{SB6} = V_{DS1}+V_{R_S}+V_{DS3} = 2.525V \notag\\
&V_{th6} = 1.18V \notag\\
&V_{GS6}=1.326V \notag\\
&V_{G6} = V_{GS6}+V_{DS1}+V_{R_S}+V_{DS3} = 3.855V \notag
\end{align}
All short-channel effect that produce a more complex threshold voltage dependence are not taken into account here.
\column[]{0.4\textwidth}
\begin{figure}
\scalebox{0.48}{
\begin{circuitikz}
\ctikzset{tripoles/mos style/arrows,bipoles/length=1cm}

\draw (-2,4) -- (-3,4)
to[Tnmos,n=M6] (M6.drain) to[R,l=$R_L$] (-3,7) -- (3,7)
(M6.source) node[right=3mm, above=3mm]{$M6$};

\draw (-2,4) -- (-1,4) to[Tnmos,mirror,n=M7] (M7.drain) to[R,l=$R_L$] (-1,7)
(M7.source) node[left=3mm, above=4mm]{$M7$};

\draw (2,4) -- (1,4) to[Tnmos,n=M8] (M8.drain) to[R,l=$R_L$] (1,7)
(M8.source) node[right=3mm, above=3mm]{$M8$};

\draw (2,4) -- (3,4) to[Tnmos,mirror,n=M9] (M9.drain) to[R,l=$R_L$] (3,7)
(M9.source) node[left=3mm, above=3mm]{$M9$};

%drawing VLO-
\draw (M7.gate) -- (M8.gate);
\draw (M7.gate) -| (0,4) to[short,-*] (0,4);

%Out nodes
\draw (M6.gate) -| (-5,4.65)  to[short,-*] (-5,4.65);
\draw (M9.gate) -| (5,4.65)  to[short,-*] (5,4.65);
\draw (-2,3) -- (-2,4)
(2,3) -- (2,4);
\draw (0,7) node[above]{$V_{DD}$};
\draw (-2,0) node[sground]{} to[open, v^=$V_{SB6}$] (M6.source) to[open, v^=$V_{GS6}$] (M6.gate);
\draw (-2.5,4) to[open,v=$V_{DS6}$] (-2.5,5);
\draw (-4,5) to[open,v^=$V_{RL}$] (-4,7);
\end{circuitikz}
}
\caption{Gain stage}
\label{fig:Mixing stage body}
\end{figure}
\end{columns}
\end{frame}


\subsection{Design by simulation}

\begin{frame} % display current subsection in subindex
\tableofcontents[currentsubsection]
\end{frame}

\begin{frame}
	\frametitle{Gilbert cell CAD design - Introduction}
	A new design approach, based on \textbf{simulation and characterization of each device} is required, since:
	\begin{itemize}
		\item Level 1 model simulation result proven to be not enough accurate to correctly describe the behaviour of the circuit (no correspondence between calculation and simulation);
		\item Gate width suggested in literature by:
		\begin{equation}
			W_{opt}=\frac{1}{3\omega L C_{ox} R_{g}} \notag
		\end{equation}
		yield too large devices (W$\geq$1mm).
	\end{itemize}
\end{frame}

%new frame
\begin{frame}
\frametitle{Gilbert cell CAD design - Step 1}
	\begin{columns}[c]
		\column{0.6\textwidth}
			\textbf{Step 1}: enhance $g_{m3}$, linearity and noise. We \textbf{impose}:
			\begin{itemize} 
				\item M\textsubscript{3,4} dimensions:
					\begin{gather}
						L_3=3 L_{min} = 1.8\mu m \notag\\
						50 \mu m \le W_3 \le 500 \mu m \notag
					\end{gather}
				\item Circuit consumption: $I_0 \approx 5mA$;
				\item Voltage drops on M\textsubscript{3,4} nodes:
					\begin{align}
						V_{SB3} &= V_{DS1} = V_{DS3} = 1.5 V \notag 
					\end{align}
				\end{itemize}
	
		\column{0.35\textwidth}
		\begin{figure}[H]
			\centering
			\scalebox{0.5}{
				\begin{circuitikz}
					\ctikzset{tripoles/mos style/arrows,bipoles/length=1cm}
					%I_0 and RS
					\draw (0,-1) node[sground]{};
					\draw (0,-1) to[Tnmos,n=M1] (0,0);
					\draw (M1.gate) to[short,-*] (M1.gate);
					\draw (0,0) to[short,i_<=$I_0$] (0,1);
					\draw (0,1) to[short,-] (-0.2,1) to[R,l_=$R_S$] (-2,1) -| (-2.5,2) to[Tnmos,n=M3] (-2.5,3) to[short,-] (-2.5,3.5) to[twoport,l=$LO_{stage}$] (-2.5,4.5);
					\draw (M3.source) node[right=3mm, above=3mm]{$M3$};
					\draw (0,1) to[short,-] (0.2,1) to[R,l=$R_S$] (2,1) -| (2.5,2) to[Tnmos,n=M4,mirror] (2.5,3) to[short,-] (2.5,3.5) to[twoport,l=$LO_{stage}$] (2.5,4.5);
					\draw (M4.source) node[left=3mm, above=3mm]{$M4$};
					\draw (M3.gate) -| (-4,2.5) to[short,-*] (-4,2.5) node[below]{};
					\draw (M4.gate) -| (4,2.5) to[short,-*] (4,2.5) node[below]{};
					%voltage arrows
					\draw (M1.source) to[open, v^=$V_{GS1}$] (M1.gate);
					\draw (-1.5,-1.7) to[open, v^=$V_{SB3}$] (-2.5,1);
					\draw (M3.source) to[open, v^=$V_{GS3}$] (M3.gate);
					\draw (-2,1.7) to[open, v=$V_{DS3}$] (-2,3);
					\draw (0,0.9) to[open, v^=$V_{RS}$] (-2,0.9);
				\end{circuitikz}
			}
			\caption{Gain stage.}
			\label{fig:GainStage}
		\end{figure}
	\end{columns}
\end{frame}

% new frame
\begin{frame}
	\frametitle{Gilbert cell CAD design - Step 1}
	\begin{figure}[H]
		\centering
		\includegraphics[scale=0.5]{W2_id}
		\caption{I\textsubscript{D3} vs V\textsubscript{GS3} curve. W\textsubscript{3} varying from 50\(\mu\)m to 500\(\mu\)m, $V_{DS3} = 1.5 V$.}
		\label{fig:W_2_id}
	\end{figure}
\end{frame}

\begin{frame}
	\frametitle{Gilbert cell CAD design - Step 1}
	 \begin{columns}[c]
	 \column{0.7\textwidth}
	 \begin{figure}[H]
	 	\centering
	 	\includegraphics[scale=0.5]{W2_gm}
	 	\caption{g\textsubscript{m3} vs V\textsubscript{GS3} curve. W\textsubscript{3} varying from 50\(\mu\)m to 500\(\mu\)m, $V_{DS3} = 1.5 V$.}
	 	\label{fig:W_2_gm}
	 \end{figure}
 	 \column{0.3\textwidth}
 	 We chose:
 	 \begin{itemize}
 	 	\item $W_3=500\mu $m
 	 	\item $V_{GS3}=1.5 $V
 	 	\item $g_{m3}=11.9$mS
 	 	\item $I_0/2=2.9$mA
 	 \end{itemize}
	 \end{columns}
	
\end{frame}

%new frame
\begin{frame}
\frametitle{Gilbert cell CAD design - Step 2}
\begin{columns}[c]
	\column{0.6\textwidth}
	
	\textbf{Step 2}: design the current sink. We \textbf{have}:
	\begin{itemize} 
		\item voltage on R\textsubscript{S}:
		\begin{gather}
		V_{R_S}=2.9mA\times 10 \Omega =29 mV \notag
		\end{gather}
		\item Voltages on M\textsubscript{1}
		\begin{align}
		V_{DS1} &=1.5 V- V_{R_S} = 1.471V\notag \\
		 V_{GS1} &= 1.471V \notag
		\end{align}
	\end{itemize}
	From simulation we get W\textsubscript{3}=373$\mu m$ to have I\textsubscript{0}=5.8mA.
	
	\column{0.35\textwidth}
	\begin{figure}[H]
		\centering
		\scalebox{0.5}{
			\begin{circuitikz}
				\ctikzset{tripoles/mos style/arrows,bipoles/length=1cm}
				%I_0 and RS
				\draw (0,-1) node[sground]{};
				\draw (0,-1) to[Tnmos,n=M1] (0,0);
				\draw (M1.gate) to[short,-*] (M1.gate);
				\draw (0,0) to[short,i_<=$I_0$] (0,1);
				\draw (0,1) to[short,-] (-0.2,1) to[R,l_=$R_S$] (-2,1) -| (-2.5,2) to[Tnmos,n=M3] (-2.5,3) to[short,-] (-2.5,3.5) to[twoport,l=$LO_{stage}$] (-2.5,4.5);
				\draw (M3.source) node[right=3mm, above=3mm]{$M3$};
				\draw (0,1) to[short,-] (0.2,1) to[R,l=$R_S$] (2,1) -| (2.5,2) to[Tnmos,n=M4,mirror] (2.5,3) to[short,-] (2.5,3.5) to[twoport,l=$LO_{stage}$] (2.5,4.5);
				\draw (M4.source) node[left=3mm, above=3mm]{$M4$};
				\draw (M3.gate) -| (-4,2.5) to[short,-*] (-4,2.5) node[below]{};
				\draw (M4.gate) -| (4,2.5) to[short,-*] (4,2.5) node[below]{};
				%voltage arrows
				\draw (M1.source) to[open, v^=$V_{GS1}$] (M1.gate);
				\draw (-1.5,-1.7) to[open, v^=$V_{SB3}$] (-2.5,1);
				\draw (M3.source) to[open, v^=$V_{GS3}$] (M3.gate);
				\draw (-2,1.7) to[open, v=$V_{DS3}$] (-2,3);
				\draw (0,0.9) to[open, v^=$V_{RS}$] (-2,0.9);
			\end{circuitikz}
		}
		\caption{Gain stage.}
		\label{fig:GainStage}
	\end{figure}
\end{columns}
\end{frame}

\begin{frame}
	\frametitle{Gilbert cell CAD design - Step 2}
	\begin{figure}[H]
		\centering
		\includegraphics[scale=0.5]{W1_id}
		\caption{I\textsubscript{D1} vs V\textsubscript{GS1} curve. W=\(373\mu m\)}
		\label{W1_id}
	\end{figure}
\end{frame}

%new frame
\begin{frame}
\frametitle{Gilbert cell CAD design - Step 3}
\begin{columns}[c]
	\column{0.65\textwidth}

	\textbf{Step 3}: design the mixing stage. 

	From \textbf{design spec} we have: 
	\begin{align}
	A_v \approx \frac{2}{\pi}\left( \frac{R_L}{R_S + \frac{1}{g_{m3}}}\right)=4 \nonumber
	\end{align}
	therefore
	\begin{align}
	R_L&=A_v \cdot \left( \frac{\pi}{2} \cdot\frac{1}{g_{m3}} + R_S \right)  \notag \\
	&= 4\times \left( \frac{\pi}{2}\times\frac{1}{11.9mS} + 10\Omega \right)=577 \Omega \notag
	\end{align}
	\column{0.35\textwidth}
		\begin{figure}[H]
			\centering
			\scalebox{0.4}{
					\begin{circuitikz}
					\ctikzset{tripoles/mos style/arrows,bipoles/length=1cm}
					
					\draw (-2,4) -- (-3,4)
					to[Tnmos,n=M6] (M6.drain) to[R,l=$R_L$] (-3,7) -- (3,7)
					(M6.source) node[right=3mm, above=3mm]{$M6$};
					
					\draw (-2,4) -- (-1,4) to[Tnmos,mirror,n=M7] (M7.drain) to[R,l=$R_L$] (-1,7)
					(M7.source) node[left=3mm, above=4mm]{$M7$};
					
					\draw (2,4) -- (1,4) to[Tnmos,n=M8] (M8.drain) to[R,l=$R_L$] (1,7)
					(M8.source) node[right=3mm, above=3mm]{$M8$};
					
					\draw (2,4) -- (3,4) to[Tnmos,mirror,n=M9] (M9.drain) to[R,l=$R_L$] (3,7)
					(M9.source) node[left=3mm, above=3mm]{$M9$};
					
					%drawing VLO-
					\draw (M7.gate) -- (M8.gate);
					\draw (M7.gate) -| (0,4) to[short,-*] (0,4);
					
					%Out nodes
					\draw (M6.gate) -| (-5,4.65)  to[short,-*] (-5,4.65);
					\draw (M9.gate) -| (5,4.65)  to[short,-*] (5,4.65);
					\draw (-2,3) -- (-2,4)
					(2,3) -- (2,4);
					\draw (0,7) node[above]{$V_{DD}$};
					\draw (-2,0) node[sground]{} to[open, v^=$V_{SB6}$] (M6.source) to[open, v^=$V_{GS6}$] (M6.gate);
					\draw (-2.5,4) to[open,v=$V_{DS6}$] (-2.5,5);
					\draw (-4,5) to[open,v^=$V_{RL}$] (-4,7);
				\end{circuitikz}
			}
			\caption{Gain stage.}
			\label{fig:GainStage}
		\end{figure}
	\end{columns}
\end{frame}

%new frame
\begin{frame}
	\frametitle{Gilbert cell CAD design - Step 3}
	\begin{columns}[c]
	\column{0.65\textwidth}
	One has:
	\begin{align}
	V_{SB6}&=V_{DS1}+V_{R_S}+V_{DS3} \notag\\
	&= 1.47V+0.029V+1.5V=3V \nonumber
	\end{align}
	hence:
	\begin{align}	
	&V_{R_L}=2.9mA\times 577\Omega=1.673V \nonumber \\
	&V_{DS6}=V_{dd}-V_{R_L}-V_{SB6}=327mV \nonumber
	\end{align}
	We need \emph{switches} slightly above threshold.
	\column{0.35\textwidth}
	\begin{figure}[H]
		\centering
		\scalebox{0.4}{
			\begin{circuitikz}
				\ctikzset{tripoles/mos style/arrows,bipoles/length=1cm}
				
				\draw (-2,4) -- (-3,4)
				to[Tnmos,n=M6] (M6.drain) to[R,l=$R_L$] (-3,7) -- (3,7)
				(M6.source) node[right=3mm, above=3mm]{$M6$};
				
				\draw (-2,4) -- (-1,4) to[Tnmos,mirror,n=M7] (M7.drain) to[R,l=$R_L$] (-1,7)
				(M7.source) node[left=3mm, above=4mm]{$M7$};
				
				\draw (2,4) -- (1,4) to[Tnmos,n=M8] (M8.drain) to[R,l=$R_L$] (1,7)
				(M8.source) node[right=3mm, above=3mm]{$M8$};
				
				\draw (2,4) -- (3,4) to[Tnmos,mirror,n=M9] (M9.drain) to[R,l=$R_L$] (3,7)
				(M9.source) node[left=3mm, above=3mm]{$M9$};
				
				%drawing VLO-
				\draw (M7.gate) -- (M8.gate);
				\draw (M7.gate) -| (0,4) to[short,-*] (0,4);
				
				%Out nodes
				\draw (M6.gate) -| (-5,4.65)  to[short,-*] (-5,4.65);
				\draw (M9.gate) -| (5,4.65)  to[short,-*] (5,4.65);
				\draw (-2,3) -- (-2,4)
				(2,3) -- (2,4);
				\draw (0,7) node[above]{$V_{DD}$};
				\draw (-2,0) node[sground]{} to[open, v^=$V_{SB6}$] (M6.source) to[open, v^=$V_{GS6}$] (M6.gate);
				\draw (-2.5,4) to[open,v=$V_{DS6}$] (-2.5,5);
				\draw (-4,5) to[open,v^=$V_{RL}$] (-4,7);
			\end{circuitikz}
		}
		\caption{Gain stage.}
		\label{fig:GainStage}
	\end{figure}
	\end{columns}
\end{frame}

\begin{frame}
	\frametitle{Gilbert cell CAD design - Step 3}
	\begin{figure}[H]
		\centering
		\includegraphics[scale=0.5]{M6_Vth}
		\caption{Extrapolation of M6 threshold voltage from transconductance versus the \(V_{GS}\) curve. The threshold is located at \(1.27V\).}
		\label{M6_Vth}
	\end{figure}
\end{frame}

%new frame
\begin{frame}
\frametitle{Gilbert cell CAD design - Step 3}
	\begin{columns}[c]
	\column{0.65\textwidth}
	From simulation we chose V\textsubscript{od6}=60mV.
	Then:
	\begin{align}
		&V_{GS6}=V_{th6}+ V_{od6}=1.33V \nonumber
	\end{align}
	By characterizing the device:
	\begin{align}
		&W_6=170.3\mu m \nonumber\\
		&L = L_{min} = 0.6\mu m \nonumber
	\end{align} 
	Minimum gate length: fast.
	\column{0.35\textwidth}
	\begin{figure}[H]
		\centering
		\scalebox{0.4}{
			\begin{circuitikz}
				\ctikzset{tripoles/mos style/arrows,bipoles/length=1cm}
				
				\draw (-2,4) -- (-3,4)
				to[Tnmos,n=M6] (M6.drain) to[R,l=$R_L$] (-3,7) -- (3,7)
				(M6.source) node[right=3mm, above=3mm]{$M6$};
				
				\draw (-2,4) -- (-1,4) to[Tnmos,mirror,n=M7] (M7.drain) to[R,l=$R_L$] (-1,7)
				(M7.source) node[left=3mm, above=4mm]{$M7$};
				
				\draw (2,4) -- (1,4) to[Tnmos,n=M8] (M8.drain) to[R,l=$R_L$] (1,7)
				(M8.source) node[right=3mm, above=3mm]{$M8$};
				
				\draw (2,4) -- (3,4) to[Tnmos,mirror,n=M9] (M9.drain) to[R,l=$R_L$] (3,7)
				(M9.source) node[left=3mm, above=3mm]{$M9$};
				
				%drawing VLO-
				\draw (M7.gate) -- (M8.gate);
				\draw (M7.gate) -| (0,4) to[short,-*] (0,4);
				
				%Out nodes
				\draw (M6.gate) -| (-5,4.65)  to[short,-*] (-5,4.65);
				\draw (M9.gate) -| (5,4.65)  to[short,-*] (5,4.65);
				\draw (-2,3) -- (-2,4)
				(2,3) -- (2,4);
				\draw (0,7) node[above]{$V_{DD}$};
				\draw (-2,0) node[sground]{} to[open, v^=$V_{SB6}$] (M6.source) to[open, v^=$V_{GS6}$] (M6.gate);
				\draw (-2.5,4) to[open,v=$V_{DS6}$] (-2.5,5);
				\draw (-4,5) to[open,v^=$V_{RL}$] (-4,7);
			\end{circuitikz}
		}
		\caption{Gain stage.}
		\label{fig:GainStage}
	\end{figure}
	\end{columns}
\end{frame}

\begin{frame}
	\frametitle{Bias net design - M\textsubscript{5} }
	\begin{columns}
		\column{0.65\textwidth}
		Now the bias net is designed. From \textbf{spec}:
		\begin{align}                                                             V_{G1}&=1.471 V \nonumber \\  
		V_{G3}&=3 V \nonumber \\
		V_{G6}&=4.33 V \nonumber
		\end{align} 
		We chose mirroring ratio 1:1, $L_2=1.8\mu m$ (same length of M\textsubscript{1}) and \(V_{GS2}=V_{DS2}=1.471V\). To have I\textsubscript{0}=5.8mA from \textbf{simulation}:
		\begin{align}
			&W_2=373\mu m \notag
		\end{align}
		Since $V_{G5}=3V$, from \textbf{simulation}:
		\begin{align}
		&W_5=130.45\mu m \nonumber\\
		&L_5=0.6\mu m \nonumber
		\end{align}  
		\column{0.35\textwidth}
		\begin{figure} [H]
			\centering
			\scalebox{0.6}{
			\begin{circuitikz}
				\ctikzset{tripoles/mos style/arrows,bipoles/length=1cm}
				\ctikzset{bipoles/capacitor/height=0.5}
				\ctikzset{bipoles/capacitor/width=0.1}
				%M2
				\draw (0,0) to[Tnmos,mirror,n=M2] (0,2);
				\draw (M2.source) node[left=3mm,above=3mm]{$M2$};
				\draw (M2.gate)[right] |- (M2.drain);
				\draw (M2.gate) to[short,-*] (2.5,1) node[right]{to $G_1$};
				\draw (M2.source) to[short] (0,0) node[sground]{};
				\draw (0,6) -- (-2,6) to[C=$C_{1}$] (-2,5) node[sground]{};
				%M5
				\draw (M2.drain) to[Tnmos,mirror,n=M5] (0,4.5);
				\draw (M5.source) node[left=3mm,above=3mm]{$M5$};
				\draw (M5.gate)[right] |- (M5.drain);
				\draw (0,4) -- (-2,4) to[C=$C_{2}$] (-2,3) node[sground]{};
				\draw (M5.gate) to[R, l_=$R_1$] (2,2.3) to[short,-*] (2.5,2.3) node[right]{to $G_3$};
				\draw (M5.gate) to[R=$R_1$] (2,3.7) to[short,-*] (2.5,3.7) node[right]{to $G_4$};
				%R2 R4
				\draw (M5.drain) to[R=$R_2$,n=R2] (0,6.3) to[R=$R_4$] (0,7.1) to[short,-*] (0,7.5) node[above]{$V_{dd}$};
				\draw (0,5.7) to[short] (0.7,5.7) to[R,l_=$R_3$] (2,5) to[short,-*] (2.5,5) node[right]{to $G_6$,$G_9$};
				\draw (0,5.7) to[short] (0.7,5.7) to[R=$R_3$] (2,6.4) to[short,-*] (2.5,6.4) node[right]{to $G_7$,$G_8$};
			\end{circuitikz}
			}
			\caption{Reference biasing network schematic}
			\label{fig:biasNet1}
		\end{figure}
	\end{columns}	
\end{frame}

\begin{frame}
	\frametitle{Bias net design - R\textsubscript{2},R\textsubscript{4}}
	\begin{columns}
	\column{0.65\textwidth}
	Voltage drop on R\textsubscript{2} and R\textsubscript{4}:
	\begin{align}
		&R_2+R_4=\frac{V_{dd}-V_{G5}}{I_0} = 344\Omega\nonumber
	\end{align}
	Since $V_{G6}=4.33V$:
	\begin{align}
		&R_2=229\Omega \nonumber\\
		&R_4=115\Omega \nonumber
	\end{align}	
	
	Total \textbf{static power consumption} is $P = 2\times(5.8mA \times 5V) = 58mW$
	\column{0.35\textwidth}
	\begin{figure} [H]
		\centering
		\scalebox{0.6}{
			\begin{circuitikz}
				\ctikzset{tripoles/mos style/arrows,bipoles/length=1cm}
				\ctikzset{bipoles/capacitor/height=0.5}
				\ctikzset{bipoles/capacitor/width=0.1}
				%M2
				\draw (0,0) to[Tnmos,mirror,n=M2] (0,2);
				\draw (M2.source) node[left=3mm,above=3mm]{$M2$};
				\draw (M2.gate)[right] |- (M2.drain);
				\draw (M2.gate) to[short,-*] (2.5,1) node[right]{to $G_1$};
				\draw (M2.source) to[short] (0,0) node[sground]{};
				\draw (0,6) -- (-2,6) to[C=$C_{1}$] (-2,5) node[sground]{};
				%M5
				\draw (M2.drain) to[Tnmos,mirror,n=M5] (0,4.5);
				\draw (M5.source) node[left=3mm,above=3mm]{$M5$};
				\draw (M5.gate)[right] |- (M5.drain);
				\draw (0,4) -- (-2,4) to[C=$C_{2}$] (-2,3) node[sground]{};
				\draw (M5.gate) to[R, l_=$R_1$] (2,2.3) to[short,-*] (2.5,2.3) node[right]{to $G_3$};
				\draw (M5.gate) to[R=$R_1$] (2,3.7) to[short,-*] (2.5,3.7) node[right]{to $G_4$};
				%R2 R4
				\draw (M5.drain) to[R=$R_2$,n=R2] (0,6.3) to[R=$R_4$] (0,7.1) to[short,-*] (0,7.5) node[above]{$V_{dd}$};
				\draw (0,5.7) to[short] (0.7,5.7) to[R,l_=$R_3$] (2,5) to[short,-*] (2.5,5) node[right]{to $G_6$,$G_9$};
				\draw (0,5.7) to[short] (0.7,5.7) to[R=$R_3$] (2,6.4) to[short,-*] (2.5,6.4) node[right]{to $G_7$,$G_8$};
			\end{circuitikz}
		}
		\caption{Reference biasing network schematic}
		\label{fig:biasNet1}
	\end{figure}
	\end{columns}	
\end{frame}

\begin{frame}
	\frametitle{Bias net design - R\textsubscript{1,3} and C\textsubscript{1,2} }
	\begin{columns}
	\column{0.65\textwidth}
	Resistors R\textsubscript{1} and R\textsubscript{3} act as AC block:
	\begin{align}
	&R_1=R_3=30k\Omega \nonumber
	\end{align}
	Equivalent resistance seen from C\textsubscript{1}:
	\begin{align}
	R_{eq}\simeq R_2||R4||\frac{R_3}{2}=76.4 \Omega \nonumber
	\end{align}
	Pole frequency un decade before f\textsubscript{LO}:
	\begin{align}
		C_{1}\ge 10\cdot \frac{1}{2\pi \cdot R_{eq} \cdot \frac{f_{lo}}{10}} = 20.8pF \notag
	\end{align}
	From \textbf{optimization}: $C_1=C_2=25pF$.
	\column{0.35\textwidth}
	\begin{figure} [H]
		\centering
		\scalebox{0.6}{
			\begin{circuitikz}
				\ctikzset{tripoles/mos style/arrows,bipoles/length=1cm}
				\ctikzset{bipoles/capacitor/height=0.5}
				\ctikzset{bipoles/capacitor/width=0.1}
				%M2
				\draw (0,0) to[Tnmos,mirror,n=M2] (0,2);
				\draw (M2.source) node[left=3mm,above=3mm]{$M2$};
				\draw (M2.gate)[right] |- (M2.drain);
				\draw (M2.gate) to[short,-*] (2.5,1) node[right]{to $G_1$};
				\draw (M2.source) to[short] (0,0) node[sground]{};
				\draw (0,6) -- (-2,6) to[C=$C_{1}$] (-2,5) node[sground]{};
				%M5
				\draw (M2.drain) to[Tnmos,mirror,n=M5] (0,4.5);
				\draw (M5.source) node[left=3mm,above=3mm]{$M5$};
				\draw (M5.gate)[right] |- (M5.drain);
				\draw (0,4) -- (-2,4) to[C=$C_{2}$] (-2,3) node[sground]{};
				\draw (M5.gate) to[R, l_=$R_1$] (2,2.3) to[short,-*] (2.5,2.3) node[right]{to $G_3$};
				\draw (M5.gate) to[R=$R_1$] (2,3.7) to[short,-*] (2.5,3.7) node[right]{to $G_4$};
				%R2 R4
				\draw (M5.drain) to[R=$R_2$,n=R2] (0,6.3) to[R=$R_4$] (0,7.1) to[short,-*] (0,7.5) node[above]{$V_{dd}$};
				\draw (0,5.7) to[short] (0.7,5.7) to[R,l_=$R_3$] (2,5) to[short,-*] (2.5,5) node[right]{to $G_6$,$G_9$};
				\draw (0,5.7) to[short] (0.7,5.7) to[R=$R_3$] (2,6.4) to[short,-*] (2.5,6.4) node[right]{to $G_7$,$G_8$};
			\end{circuitikz}
		}
		\caption{Reference biasing network schematic}
		\label{fig:biasNet1}
	\end{figure}
	\end{columns}	
\end{frame}

\begin{frame}
	\frametitle{Design CAD validation - Full circuit}
	\begin{figure}[H]
		\centering
		\includegraphics[width=0.5\textwidth, angle=-90]{S_cad_full}
		\caption{Gilbert cell and bias network schematic}
		\label{S_cad_full}
	\end{figure}
\end{frame}

\begin{frame}
	\frametitle{Design CAD validation - Gain stage}
	\begin{figure}[H]
		\centering
		\includegraphics[width=\textwidth]{S_cad_I0}
		\caption{Close up view on current sink and RF stage}
		\label{S_cad_I0}
	\end{figure}
\end{frame}

\begin{frame}
	\frametitle{Design CAD validation - Mixing stage}
	\begin{figure}[H]
	\centering
	\includegraphics[scale=0.13]{S_cad_LO}
	\caption{Close up view of LO stage\label{subfig-1:S_cad_LO}}
	\label{S_cad_I0}
	\end{figure}
\end{frame}

\begin{frame}
	\frametitle{Design CAD validation - Bias net }
	\begin{figure}[H]
	\centering
	\includegraphics[scale=0.10]{S_cad_biasnet}
	\caption{Close up view of LO stage\label{subfig-1:S_cad_biasnet}}
	\label{S_cad_I0}
	\end{figure}
\end{frame}

\subsection{Layout of the Gilbert cell}
\begin{frame}
	\tableofcontents[currentsubsection]
\end{frame}

\begin{frame}
	\frametitle{Layout strategy}
	\begin{itemize}
		\item Common centroid, interdigitated structures: less gradients;
		\item Multi-finger structure with same-length for transistors fingers: minimize encroachment;
		\item Components with same alignment: uniform error distribution;
		\item Dummy elements: less border effects;
		\item Limited substrate noise with guard rings;
		\item Minimum number of crossed connections and metal changes in the routing process (less parasitics);
		\item Compact and symmetric structure;
		\item Multiple substrate contacts and isolating well used because of circuit width and shared body contact for each MOSFET: less substrate currents.
		\item \textbf{Every device has been optimized} to have matching between circuit and layout.  
	\end{itemize}
\end{frame}

\begin{frame}
	\frametitle{Layout - Gain and Mixing stage}
	\begin{columns}
		\column{0.6\textwidth}
		\begin{itemize}
		\item A \textbf{symmetric} input low noise differential stage is desired.
		\item A common centroid structure is employed. This minimizes \textbf{offset} voltage.
		\item Dummy elements to reduce \textbf{lateral diffusion} variations.
		\item Same approach with mixing stage: designed to be easily stacked above RF stage, with the two switching stages kept close to each other.
		\end{itemize}
		\column{0.4\textwidth}
		\begin{figure}[H]
			\centering
			\includegraphics[scale=0.3]{diff_CC}
			\caption{Common centroid structure used for differential RF stage}
			\label{fig:diff_CC}
		\end{figure}
	\end{columns}
\end{frame}

\begin{frame}
	\frametitle{Layout - Gain stage}
	\begin{columns}
		\column{0.3\textwidth}
		M\textsubscript{3,4} parameters:
			\begin{itemize}
				\item m = 18(+2)
				\item w$^*$=34.35$\mu$m
				\item W = 618.3$\mu$m (vs 500$\mu$m)
				\item L = 1.8$\mu$m
				\item y = 126$\mu$m
				\item x = 99.45$\mu$m
			\end{itemize}
		\column{0.7\textwidth}
		\begin{figure}[H]
			\centering
			\includegraphics[scale=0.19]{L_RF_full}
			\label{L_RF_full}
		\end{figure}
	\end{columns}
\end{frame}

\begin{frame}
\frametitle{Layout - Mixing stage}
	\begin{columns}
	\column{0.3\textwidth}
	M\textsubscript{6,7,8,9} parameters:
	\begin{itemize}
		\item m = 12(+2)
		\item w$^*$=17.55$\mu$m
		\item W = 210.6$\mu$m (vs 170.3$\mu$m)
		\item L = 0.6$\mu$m
		\item y = 56.85$\mu$m
		\item x = 87.6$\mu$m
	\end{itemize}
	\column{0.7\textwidth}
	\begin{figure}[H]
		\centering
		\includegraphics[width=\textwidth]{L_LO_full}
		\label{L_LO_full}
	\end{figure}
	\end{columns}
\end{frame}

\begin{frame}
	\frametitle{Layout - Current mirror}
	\begin{columns}
		\column{0.4\textwidth}
		M\textsubscript{1,2} parameters:
		\begin{itemize}
			\item m = 10(+1)
			\item w$^*$=39.15$\mu$m
			\item W = 391.15$\mu$m (vs 373$\mu$m)
			\item L = 1.8$\mu$m
			\item y = 61.5$\mu$m
			\item x = 87.75$\mu$m
		\end{itemize}
		\column{0.6\textwidth}
		\begin{figure}[H]
			\centering
			\includegraphics[width=\textwidth]{L_curmir_full}
			\label{L_curmir_full}
		\end{figure}
	\end{columns}
\end{frame}


\begin{frame}
\frametitle{Layout - Current mirror}
	\begin{itemize}
		\item Current mirror shows multi-finger interdigitated structure:\\ \textbf{junction cap} approximately halved due to sharing of drain contacts.
		\item Same alignment and dummy elements reduce gradients and \textbf{diffusion differences}.
		\item Modular unitary width: Current ration independent on \textbf{$W_{enc}$}.
		\item Multiple contacts to reduce series resistance.
	\end{itemize}
\end{frame}

\begin{frame}
	\frametitle{Layout - M\textsubscript{5}}
	\begin{columns}
		\column{0.4\textwidth}
		M\textsubscript{5} shows multifinger interdigitated structure with dummy 	elements. M\textsubscript{5} parameters:
		\begin{itemize}
			\item m = 12(+2)
			\item w$^*$=13.65$\mu$m
			\item W = 163.8$\mu$m (vs 130.45$\mu$m)
			\item L = 1.8$\mu$m
			\item y = 25.2$\mu$m
			\item x = 35.55$\mu$m
		\end{itemize}
		\column{0.6\textwidth}
		\begin{figure}[H]
			\centering
			\includegraphics[width=0.8\textwidth]{L_M5_full}
			\label{L_M5_full}
		\end{figure}
	\end{columns}
\end{frame}

\begin{frame}
\frametitle{Layout - C\textsubscript{bias}}
\begin{columns}
	\column{0.6\textwidth}
	\begin{itemize}
		\item C\textsubscript{1,2} are poly capacitors (poly1 + elec) with dummy elements in order to reduce dimensions(\textbf{G\textsubscript{ox}}) and improve tolerances. 
		\item Common centroid layout inside n-well (connected to V\textsubscript{DD}) to reduce \textbf{fringing} field leaks along with dummy elements.
		\item \textbf{Square units}: minimize random errors, same undercut keeps constant ratios.
	\end{itemize}
	\column{0.4\textwidth}
	\begin{figure}[H]
		\centering
		\includegraphics[width=1\textwidth]{Cap1}
		\caption{Bias capacitors layout structure}
		\label{Cap1}
	\end{figure}
\end{columns}
\end{frame}

\begin{frame}
\frametitle{Layout - C\textsubscript{bias}}
\begin{columns}
	\column{0.4\textwidth}
	\begin{itemize}
		\item y = 448.65$\mu$m
		\item x = 472.5$\mu$m
		\item C\textsubscript{bias}=25.9pF
	\end{itemize}
	
	\column{0.6\textwidth}
	\begin{figure}[H]
		\centering
		\includegraphics[width=1\textwidth]{L_Cb_full}
		\label{L_Cb_full}
	\end{figure}
\end{columns}
\end{frame}

\begin{frame}
\frametitle{Layout - C\textsubscript{signal}}
\begin{columns}
	\column{0.6\textwidth}
	\begin{itemize}
	\item C\textsubscript{signal} are large capacitances ($\simeq$nF).
	\item Multilayer structure impossible because of \emph{CMOS design rule 11.6}.
	\begin{figure}[H]
		\centering
		\includegraphics[width=0.5\textwidth]{Multicap}
		\label{Cap3}
	\end{figure}
	\item Good matching required  then common centroid structure is used. \item Series resistance reduced by surrounding metal plates. \item N-well ring connected to V\textsubscript{DD} to reduce fringing field.
	\end{itemize}
	\column{0.4\textwidth}
	\begin{figure}[H]
		\centering
		\includegraphics[width=1\textwidth]{Cap2}
		\caption{Common centroid structure of the signal capacitors}
		\label{Cap2}
	\end{figure}
\end{columns}
\end{frame}

\begin{frame}
	\frametitle{Layout - C\textsubscript{signal}}
	\begin{columns}
	\column{0.3\textwidth}
	\begin{itemize}
		\item y = 678.45$\mu$m
		\item x = 480.25$\mu$m
		\item C\textsubscript{bias}=90.1pF
	\end{itemize}
	\column{0.7\textwidth}
	\begin{figure}[H]
		\centering
		\includegraphics[width=0.7\textwidth]{L_Cs_full}
		\label{L_Cs_full}
	\end{figure}
	\end{columns}
\end{frame}

\begin{frame}
\frametitle{Layout - R\textsubscript{1,3}}
\begin{columns}
	\column{0.5\textwidth}
	\begin{itemize}
	\item R\textsubscript{1,3} are large, then n-well technology necessary with common centroid structure to improve gradients (precision not necessary though). 
	\item Since nearby mixing stage guard ring needed to avoid substrate currents.
	\end{itemize}
	\column{0.5\textwidth}
	\begin{figure}[H]
		\centering
		\includegraphics[scale=0.15]{L_R1R3}
		\label{L_R1R3}
	\end{figure}
\end{columns}
\end{frame}

\begin{frame}
\frametitle{Layout - R\textsubscript{2},R\textsubscript{4},R\textsubscript{L},R\textsubscript{S}}
\begin{columns}
	\column{0.5\textwidth}
	\begin{itemize}	
	\item R\textsubscript{2},R\textsubscript{4},R\textsubscript{L} and R\textsubscript{S} are smaller than R\textsubscript{1,3} therefore poly1 resistors can be used, improving also precision. 
	\item Common centroid structure with dummy elements.
	\end{itemize}
	\column{0.5\textwidth}
	\begin{figure}[H]
		\centering
		
		\subfloat[$R_2$\label{subfig-1:L_R2}]
		{\includegraphics[width=0.4\textwidth]{L_R2}}
		\subfloat[$R_4$\label{subfig-1:L_R4}]
		{\includegraphics[width=0.5\textwidth]{L_R4}}
		\vfill
		\subfloat[$R_L$\label{subfig-1:L_RL}]
		{\includegraphics[width=0.55\textwidth]{L_RL}}
		\hfil
		\subfloat[$R_S$\label{subfig-1:L_RS}]
		{\includegraphics[width=0.2\textwidth]{L_RS}}
		\caption{Poly resistors layout}
	\end{figure}
\end{columns}
\end{frame}

\begin{frame}
	\frametitle{Layout - Resistors parameters}
	\begin{table} [h]
		\label{tab:specs}
		\caption{Resistors parameters}
		\centering	
		\begin{tabular}{lccccc} 
			\toprule 
			Parameter & R\textsubscript{1,3}& R\textsubscript{2} & R\textsubscript{4}&R\textsubscript{L}&R\textsubscript{S} \\ 
			\midrule
			y [$\mu m$] &112.05&27.45&20.1&32.4&54\\
			x [$\mu m$] &66.3&37.35&37.65&63&13.35\\
			h [$\mu m$] &4.95&5.7&5.7&4.2&12.45\\
			w [$\mu m$] &46.5&26.1&26.25&48.6&4.95\\
			m			&4&2&1&1&1\\
			R$^*$ [$\Omega$] & 7545&114.5&115.1&289.3&9.94\\
			R\textsubscript{tot} [$\Omega$]&30180&229&115.1&289.3&9.94\\
			\bottomrule 
		\end{tabular}	
	\end{table}

\end{frame}

\begin{frame}
	\frametitle{Layout - Full circuit}
	Merging:
	\begin{itemize}
		\item Components placed to shorten interconnections, avoid crosstalk, emphasize symmetries to make paths equal for high frequency signals;
		\item Metal width dimension and number of vias to avoid electromigration.
		\item As many sub contacts as possible, connected to ground, to capture free charges in the substrate (less noise).
		\item Occupied area: $A\simeq0.7mm \cdot 1.8mm = 1.26 mm^2$;
		\item Large amount of area occupied by capacitors.
		\item Entire metal plates for V\textsubscript{DD} and GND could not be used (too few).
	\end{itemize}
\end{frame}

\begin{frame}
	\frametitle{Layout - Full circuit}
	\begin{figure}[H]
		\centering
		\includegraphics[width=1\textwidth]{L_complete_full}
		\label{L_complete_full}
	\end{figure}
\end{frame}

\begin{frame}
	\frametitle{Layout - Full circuit without capacitors}
	\begin{figure}[H]
		\centering
		\includegraphics[scale=0.18]{L_complete_zoom}
		\label{L_complete_zoom}
	\end{figure}
\end{frame}